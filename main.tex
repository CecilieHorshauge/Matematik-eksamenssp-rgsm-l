\documentclass[a4paper, 11pt]{article}

\usepackage[danish]{babel}
\usepackage[utf8]{inputenc}
%\usepackage{tgtermes}
%\usepackage{fouriernc}
\usepackage[T1]{fontenc}
\usepackage[margin=3cm]{geometry}

\usepackage{graphicx}
\usepackage{amssymb}
\usepackage{amsmath}
\usepackage{amsthm}
\usepackage{multicol}
\usepackage{xcolor}
\usepackage{wrapfig}
\usepackage{array} %skema

\usepackage{enumerate}
\usepackage[shortlabels]{enumitem}
\usepackage{verbatim}
\usepackage{hyperref}
\hypersetup{
    colorlinks=true,
    linkcolor=red,   
    urlcolor=red,
}
\newcommand{\N}{\mathbb{N}}
\newcommand{\Z}{\mathbb{Z}}
\newcommand{\Q}{\mathbb{Q}}
\newcommand{\R}{\mathbb{R}}

%Is defined to be equal to
\newcommand*{\defeq}{\mathrel{\vcenter{\baselineskip0.5ex \lineskiplimit0pt
                     \hbox{\scriptsize.}\hbox{\scriptsize.}}}%
                     =}
 
\title{{\large \textsc{MATEMATIK A\\ eksamensspørgsmål}}} 
\author{Cecilie Horshauge}
\date{\today}

\begin{document}
\maketitle
% Intro til opgaven
\noindent 

\section*{Spørgsmål 1} 
\begin{enumerate}[label=\alph*)]
    \item Redegør for løsning af spørgsmål 3 i projekt afkøling. (differentialligninger)
    \item Redegør for andengradspolynomier på formen \(f(x)=ax^2+b+c\)
\end{enumerate}
\indent \indent Kom herunder ind på:
\begin{itemize}
    \item Konstanternes betydning
    \item Grafens udseende
    \item rødder (andengradsligning)
    \item toppunkt
\end{itemize}
\subsection*{Gennemgang}
\underline{Konstanternes betydning + grafens udseende}\\
\(a>0\): Glad parabel\\
\(a<0\): Sur parabel\\
Brug den afledte funktion til ar argumentere for det. \\\\
\(f'(x)=2ax+b\)\\
\(f'(0)=b\), dvs. at tangenthældingen ved \(x=0\) er \(b\).\\\\
Skæring med y-aksen ved \(x=0\) er lig med \(c\).\\\\
\underline{Rødder}
Rødderne findes ved \(y=0\), vi bestemmer dem ved at løse ligningen \(ax^2+bx+c=0\).
\begin{align}
    ax^2+bx+c&=0\\
    4a^2x^2+4abx+4ac&=0\\
    4a^2x^2+4abx&=-4ac\\
    4a^2x^2+4abx+b^2&=-4ac+b^2\\
    (2ax+b)^2&=-4ac+b^2\\
    2ax+b&= \pm \sqrt{b^2-4ac}\\
    2ax &= -b \pm \sqrt{b^2-4ac}\\
    x &= \frac{-b \pm \sqrt{b^2-4ac}}{2a}
\end{align}
\underline{Toppunkt}\\
I Toppunktet er tangenthældingen lig med 0. Vi havde tidligere beregnet \(f'(x)=2ax+b\).
\section*{Opgave 2}
\begin{enumerate}[label=\alph*)]
    \item Redegør for løsning af spørgsmål 3 i projekt drikkebæger. (Integralregning)
    \item Redegør for trigonometri på vilkårlige trekanter.
\end{enumerate}
\indent \indent Kom herunder ind på:
\begin{itemize}
    \item Areal af en vilkårlig trekant
    \item sinusrelationen
    \item cosinusrelationen
    \item Indskrevne og omskrevne cirkler
\end{itemize} 
\clearpage
\section*{Opgave 3}
\begin{enumerate}[label=\alph*)]
    \item Redegør for løsning af spørgsmål 4) i projekt ”afkøling”. (differentialligninger)
    \item Redegør for vektorer i planen.
\end{enumerate}
\indent \indent Kom herunder ind på:
\begin{itemize}
    \item regning med vektorer (sum, differens mv.)
    \item skalarprodukt
    \item vinkel mellem vektorer
    \item vektorer i polære koordinater
    \item vektorprojektion
    \item areal af det udspændte parallelogram
\end{itemize} 
\subsection*{Gennemgang}
\underline{Spørgsmål 4 i projekt afkøling}
\clearpage
\section*{Opgave 4}
\begin{enumerate}[label=\alph*)]
    \item Redegør for løsning af spørgsmål 4 i projekt drikkebæger. (Integralregning)
    \item Redegør for trigonometri på vilkårlige trekanter.
\end{enumerate}
\indent \indent Kom herunder ind på:
\begin{itemize}
    \item Afstand mellem to punkter i planen
    \item Linjens ligning på normalform
    \item Afstand mellem punkt og linje i planen
    \item Cirklens ligning
    \item Ortogonale linjer
\end{itemize}
\subsection*{Gennemgang}
\underline{Løsning af spørgsmål 4 i projekt drikkebæger}\\\\
Projekt 
\section*{Opgave 5}
 
\section*{Spørgsmål 7} 
\begin{enumerate}[label=\alph*)]
    \item Redegør for løsning af spørgsmål e i projekt avedøreværket. (rumgeometri)
    \item Redegør for differentialregning.
\end{enumerate}
\indent \indent Kom herunder ind på:
\begin{itemize}
    \item Definition af differentialkvotient
    \item Tretrinsmetoden 
    \item Regneregler for differentiable funktioner
\end{itemize}
John foreslår, at for at spare tid redegører man for tretrinsmetoden med at eksempel på regneregler.
\end{document}